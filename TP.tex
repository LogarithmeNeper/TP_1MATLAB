\documentclass[11pt]{beamer}
\mode<presentation>{\usetheme{CambridgeUS}}
\usefonttheme{serif}

\usepackage[utf8]{inputenc}
\usepackage{graphicx}
\usepackage{booktabs}
\usepackage[frenchb]{babel}
\usepackage{wrapfig}
\usepackage{tikz}
\usepackage{listings}
\usetikzlibrary{shapes}
\usepackage{verbatim}
\newcommand{\verbatimfont}[1]{\renewcommand{\verbatim@font}{\ttfamily#1}}
\usepackage[hang,small]{caption}

\newcommand{\muskern}{\kern-.15ex }

\newcommand{\ppp}{\textnormal{\textbf{\textit{p\muskern p\muskern p}}}}

\lstset{language={Python}}
\usepackage{mathrsfs}

\newtheorem{defini}{Définition}[section]
\newtheorem{corollaire}{Corollaire}[section]
\newtheorem{theoreme}{Théorème}[section]
\newtheorem{proposition}{Proposition}[section]
\newtheorem{lemme}{Lemme}[section]
\newtheorem{lemmule}{Lemmule}[section]
\newtheorem{qouverte}{Question ouverte}[section]
\newtheorem{remarque}{Remarque}[section]
\newtheorem{loi}{Loi}[section]

\newenvironment{preuve}{
	\noindent\emph{Preuve.}
}

\newenvironment{exemple}{
	\noindent\emph{Exemple.}
}

\newenvironment{idpreuve}{
	\noindent\emph{Idée de la démonstration.}
}

\newenvironment{protopreuve}{
	\noindent\emph{Prototype de la démonstration.}
}

\newcommand{\dd}{
	\mathop{}\mathopen{}\mathrm{d}
}

\newcommand{\sgn}{
	\mathop{}\mathopen{}\mathrm{sgn}
}

\newcommand{\diam}{
	\mathop{}\mathopen{}\mathrm{diam}
}

\title[Calcul Matriciel]{Calcul matriciel : TP n$^\circ$1}

\author{Binôme n$^\circ$3109}
\date{\today}

\begin{document}
	\begin{frame}
		\titlepage
	\end{frame}
	
	\begin{frame}
		\tableofcontents
	\end{frame}

	\section{Première partie : équation de la chaleur}
	\begin{frame}
		\tableofcontents[currentsection]
	\end{frame}

	\begin{frame}
		\frametitle{Méthodes itératives}
		\begin{itemize}
			\item Traduction des éléments donnés en énoncé dans le langage MATLAB.
			\item[$>$] \begin{semiverbatim}
				relaxation.m, gauss\_seidel.m, jacobi.m, omega\_optimal.m
			\end{semiverbatim}
			\item On utilise ces méthodes seulement si les conditions d'application des théorèmes sous-jacents sont remplis. 
			\item[$>$] \begin{semiverbatim}
				diag\_dom.m, rayon\_spectral.m
			\end{semiverbatim}
		\end{itemize}
	\end{frame}

	\begin{frame}
		\frametitle{Implémentation mathématique}
		Pour les méthodes, on se fond sur les formules suivantes ($AX=B$) : 
		\begin{proposition}[Gauss-Siedel]
			\[X_i^{(m+1)}=\frac{1}{[A]_{i,i}}\left(B_i - \sum_{j=1}^{i-1} [A]_{i,j} X_j^{(m+1)} - \sum_{j=i+1}^{n} [A]_{i,j} X_j^{(m)}\right)\]
		\end{proposition}
	
		\begin{proposition}[Jacobi]
			\[X_i^{(m+1)}=\frac{1}{[A]_{i,i}}\left(B_i - \sum_{\substack{j=1 \\ j \neq i}}^{n} [A]_{i,j} X_j^{(m)} \right)\]
		\end{proposition}
	\end{frame}

	\begin{frame}
		\frametitle{Implémentation mathématique}
		\begin{proposition}[Relaxation]
			Si jamais on a $0<\omega<2$ et dans ce cas seulement, alors :
			\[X_i^{(m+1)}=X_i^{(m)}+\frac{\omega}{[A]_{i,i}}\left(B_i - \sum_{j=1}^{i-1} [A]_{i,j} x_j^{(m+1)} - \sum_{j=i+1}^{n} [A]_{i,j} X_j^{(m)}\right)\]
		\end{proposition}
	\end{frame}
	\begin{frame}
		\frametitle{Application à l'équation de la chaleur}		
		\[\frac{\partial T}{\partial t}=D.\nabla^2 T\]
		
		On utilise les éléments finis et un développement de Taylor pour modéliser la grille avec les conditions (raccord, bords).
		
		Deux parties distinctes :
		\begin{itemize}
			\item Équilibre : on part d'une configuration et on fait évoluer jusqu'à atteindre l'équilibre.
			\item Chauffage : on part d'une configuration puis on chauffe en continu.
		\end{itemize}
	
		Dans les deux cas : pas d'effets de bord (voisins fictifs : modification du laplacien en conséquence)
	\end{frame}

	\begin{frame}
		\frametitle{Mise en pratique (1)}
		\begin{itemize}
			\item Génération : dépend du nombre de voisins (qui dépend du lieu) $>$ remplissage de la matrice point par point, et choix des conditions de l'énoncé (bords à 100$^\circ$, points chauds à 500$^\circ$)
			\item Trace : utilisation de la fonction \emph{surf}.
		\end{itemize}
		\begin{center}
			\includegraphics[width=5cm]{plateau.png}
		\end{center}
	\end{frame}
	
	\begin{frame}
		\frametitle{Mise en pratique (1)}
		\begin{itemize}
			\item Génération : \textbf{dépend du nombre de voisins} 
		\end{itemize}
		\begin{center}
			\includegraphics[width=8cm]{voisins.png}
		\end{center}
	\end{frame}
	\begin{frame}
		\frametitle{Mise en pratique (2)}
		Puis :
		\begin{itemize}
			\item On fait une convergence soit :
			\[(U_n)_n\to U_{eq} : \forall\varepsilon>0, \exists N\in\mathbb{N}, \forall n\geq N, \underset{dU}{\underbrace{||U_n-U_{eq}||}}<\varepsilon\]
			\item Or il faut fixer et chercher le rang $N$ (éviter les répétitions inutiles), ici $\varepsilon=10^{-8}$.
			\item Exponentielle de matrice\footnote{On a un terme, écrit sous forme simplifiée $\frac{dU}{U}(t)$ que l'on intègre. Sort un $\log$ donc une exponentielle de matrice} : correspond à l'équation différentielle considérée.
		\end{itemize}
\end{frame}

	\section{Seconde partie : équation des ondes}

	\begin{frame}
		\tableofcontents[currentsection]
	\end{frame}

	\begin{frame}
		\frametitle{Méthode générale}
		\begin{itemize}
			\item On réutilise les méthodes mises en \oe uvre dans la première partie mais un peu différent.
			\item On considère un tambour modifié. 
			\item Équation vue comme un opérateur.
			\item Différences finies, permettent de chercher les modes propres (cf. diagonalisation de l'opérateur hamiltonien $\hat{H}$ en MQ, corde de Melde en mécanique)
		\end{itemize}	
	\end{frame}

	\begin{frame}
		\frametitle{Implémentation mathématique}
		\begin{proposition}[Méthode de la puissance itérée]
			Sous les hypothèses :
			\[X_k=\frac{Y_k}{||Y_k||}\]
			\[Y_{k+1}=AX_k\]
			convergent vers la bonne valeur propre.
		\end{proposition}
	
		\begin{proposition}[Weilandt]
			Si $\sigma(M)=\{\lambda_1,...,\lambda_n\}$ avec $U_1,...U_n$ et $V_1,...,V_n$ ve.p. resp. g et d. :
			\[M-\lambda_1\frac{V_1U_1}{U_1V_1}\]
			(Dimensions OK, division par un réel et $\mathcal{M}_n(\mathbb{R}$ sinon).
		\end{proposition}
	\end{frame}

	\begin{frame}
		\frametitle{Implémentation mathématique}
		\begin{lemme}
			On peut adapter avec $A^{-1}$ (en supposant que...) et on trouve alors l'inverse (donc petit module)
		\end{lemme}
	
		\begin{proposition}[Décomposition QR]
			Si $A=QR$ ($Q\in\mathcal{O}_n(\mathbb{R})$ unitaire, $R$ trig. sup.), alors on peut prendre :
			\[A_{n+1}=Q^{-1}A_nQ\]
			Idem pour LU.
		\end{proposition}
		$\to$ Valeurs singulières.
	\end{frame}

		\begin{frame}
		\frametitle{Mise en pratique}
		\begin{itemize}
			\item Initialisation de la grille (600$\times$600) sur le même principe : particularité du barreau central (voisin fictif) et des parties libres.
			\item Besoin de plus de soin dans la partie mathématique : utilisation d'une puissance itérée avec déflation de \textsc{Weilandt}. 
		\end{itemize}
	\end{frame}

	\begin{frame}
		\frametitle{Autour du bonus}
	\end{frame}
\end{document}